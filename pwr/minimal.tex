\documentclass[aspectratio=169]{beamer}
\usepackage{polski}
\usepackage[utf8]{inputenc}
\usepackage[T1]{fontenc}

\usetheme[lang=pl,hr=true]{NewPwr}

\author{Krystian Klimek}
\title{Automatyzacja dokumentacji technicznej}
\subtitle{Wykorzystanie narzędzi Sphinx oraz LaTeX (Beamer)}
\institute{Politechnika Wrocławska}
\date{\today}

\begin{document}


\begin{frame}
 \maketitle
\end{frame}

\begin{frame}{Przeźrocze techniczne}
 \begin{columns}
  \begin{column}{0.5\textwidth}
   \begin{block}{Informacje o zadaniu}
    \begin{itemize}
     \item \textbf{Kurs:} Technologie Informacyje
     \item \textbf{Prowadzący:} [Piotr Czaja]
     \item \textbf{Student:} Krystian Klimek
     \item \textbf{Zadanie:} Prezentacja Beamer
    \end{itemize}
   \end{block}
  \end{column}
  \begin{column}{0.4\textwidth}
   \centering
   \textit{Zadanie ILAAN01 Pierwsza Prezentacja w Pakiecie Beamer}
  \end{column}
 \end{columns}
\end{frame}

\begin{frame}{Zaliczenie zadania}
\begin{itemize}
    \item Link do Repozytorium Git: https://github.com/Kerry4325/ILAAN01BeamerPrezentacja.git
\end{itemize}
\end{frame}

\begin{frame}{Plan prezentacji}
 \tableofcontents
\end{frame}


\section{Narzędzie Sphinx}
\begin{frame}{System Sphinx i reStructuredText}
 \begin{itemize}
  \item \textbf{Sphinx} to generator dokumentacji pierwotnie stworzony dla Pythona.
  \item Wykorzystuje lekki język znaczników \texttt{reStructuredText} (rst).
  \item Pozwala na eksport do wielu formatów:
  \begin{itemize}
   \item Strony HTML (interaktywne).
   \item Pliki PDF (poprzez LaTeX).
   \item ePub (dla czytników e-booków).
  \end{itemize}
 \end{itemize}
\end{frame}

\section{Integracja z LaTeX}
\begin{frame}{Dlaczego LaTeX i Beamer?}
 \begin{block}{Zalety rozwiązania}
  \begin{itemize}
   \item Pełna kontrola nad typografią i składem matematycznym.
   \item Możliwość ponownego wykorzystania fragmentów kodu z raportów w prezentacjach.
   \item Profesjonalny wygląd zgodny z Identyfikacją Wizualną PWr.
  \end{itemize}
 \end{block}
\end{frame}

\section{Podsumowanie}
\begin{frame}{Podsumowanie}
 \begin{itemize}
  \item Opanowanie pakietu Beamer pozwala na szybkie tworzenie spójnych prezentacji.
  \item Narzędzia typu "Text-to-Document" (Sphinx/LaTeX) są kluczowe w pracy inżyniera.
  \item Prezentacja stanowi dopełnienie dokumentacji technicznej przygotowanej wcześniej.
 \end{itemize}
\end{frame}

\begin{frame}{Materiały źródłowe}
 \begin{thebibliography}{10}
  \bibitem{sphinx} Dokumentacja Sphinxa, \url{www.sphinx-doc.org}
  \bibitem{pwr} System Identyfikacji Wizualnej PWr (Szablon NewPwr).
  \bibitem{beamer} User Guide to the Beamer Class, T. Tantau.
 \end{thebibliography}
\end{frame}

\begin{frame}
 \centering
 \Huge \textbf{Dziękuję za uwagę!}
 
 \vspace{20pt}
 \normalsize
 Krystian Klimek \\
\end{frame}

\end{document}